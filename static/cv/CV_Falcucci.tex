%!TEX TS-program = xelatex
%!TEX encoding = UTF-8 Unicode
% Awesome CV LaTeX Template for CV/Resume
%
% This template has been downloaded from:
% https://github.com/posquit0/Awesome-CV
%
% Author:
% Claud D. Park <posquit0.bj@gmail.com>
% http://www.posquit0.com
%
%
% Adapted to be an Rmarkdown template by Mitchell O'Hara-Wild
% 23 November 2018
%
% Template license:
% CC BY-SA 4.0 (https://creativecommons.org/licenses/by-sa/4.0/)
%
%-------------------------------------------------------------------------------
% CONFIGURATIONS
%-------------------------------------------------------------------------------
% A4 paper size by default, use 'letterpaper' for US letter
\documentclass[11pt,a4paper,]{awesome-cv}

% Configure page margins with geometry
\usepackage{geometry}
\geometry{left=1.4cm, top=.8cm, right=1.4cm, bottom=1.8cm, footskip=.5cm}


% Specify the location of the included fonts
\fontdir[fonts/]

% Color for highlights
% Awesome Colors: awesome-emerald, awesome-skyblue, awesome-red, awesome-pink, awesome-orange
%                 awesome-nephritis, awesome-concrete, awesome-darknight

\definecolor{awesome}{HTML}{009ACD}

% Colors for text
% Uncomment if you would like to specify your own color
% \definecolor{darktext}{HTML}{414141}
% \definecolor{text}{HTML}{333333}
% \definecolor{graytext}{HTML}{5D5D5D}
% \definecolor{lighttext}{HTML}{999999}

% Set false if you don't want to highlight section with awesome color
\setbool{acvSectionColorHighlight}{true}

% If you would like to change the social information separator from a pipe (|) to something else
\renewcommand{\acvHeaderSocialSep}{\quad\textbar\quad}

\def\endfirstpage{\newpage}

%-------------------------------------------------------------------------------
%	PERSONAL INFORMATION
%	Comment any of the lines below if they are not required
%-------------------------------------------------------------------------------
% Available options: circle|rectangle,edge/noedge,left/right

\name{Armando}{Falcucci}

\position{Postdoctoral Researcher}
\address{Department of Geosciences, Early Prehistory and Quaternary
Ecology, Eberhard Karls University of Tübingen, Germany}

\email{\href{mailto:armando.falcucci@uni-tuebingen.de}{\nolinkurl{armando.falcucci@uni-tuebingen.de}}}
\homepage{armandofalcucci.com}
\orcid{0000-0002-3255-1005}
\googlescholar{hsrY\_B4AAAAJ}
\github{ArmandoFalcucci}

% \gitlab{gitlab-id}
% \stackoverflow{SO-id}{SO-name}
% \skype{skype-id}
% \reddit{reddit-id}


\usepackage{booktabs}

\providecommand{\tightlist}{%
	\setlength{\itemsep}{0pt}\setlength{\parskip}{0pt}}

%------------------------------------------------------------------------------


\usepackage{float}

% Pandoc CSL macros

\begin{document}

% Print the header with above personal informations
% Give optional argument to change alignment(C: center, L: left, R: right)
\makecvheader

% Print the footer with 3 arguments(<left>, <center>, <right>)
% Leave any of these blank if they are not needed
% 2019-02-14 Chris Umphlett - add flexibility to the document name in footer, rather than have it be static Curriculum Vitae
\makecvfooter
  {December 2023}
    {Armando Falcucci~~~·~~~Curriculum Vitae}
  {\thepage~ of \pageref{LastPage}~}


%-------------------------------------------------------------------------------
%	CV/RESUME CONTENT
%	Each section is imported separately, open each file in turn to modify content
%------------------------------------------------------------------------------



\hypertarget{professional-experience}{%
\section{Professional Experience}\label{professional-experience}}

\begin{cventries}
    \cventry{Postdoctoral Researcher}{Eberhard Karls University of Tübingen}{Tübingen, Germany}{Sept 2020--Present}{\begin{cvitems}
\item DFG project "Investigating Early Upper Paleolithic Technological Variability and Cultural Dynamics South of the Alps"  \href{https://gepris.dfg.de/gepris/projekt/431809858?language=en}{\faExternalLink}
\end{cvitems}}
    \cventry{Research Scholar}{Harvard University}{Cambridge, MA, USA}{March 2023--June 2023}{\begin{cvitems}
\item J-1 Short-Term Scholar at the Peabody Museum of Archaeology and Ethnology and the Department of Anthropology
\end{cvitems}}
    \cventry{Research Assistant}{Heidelberg Academy of Sciences and Humanities}{Tübingen, Germany}{Jan 2020--Aug 2020}{\begin{cvitems}
\item The Role of Culture in Early Expansions of Humans (ROCEEH)  \href{https://www.hadw-bw.de/en/research/research-center/roceeh}{\faExternalLink}
\item Synthesizing archaeological data into the ROCEEH Out of Africa Database (ROAD)
\end{cvitems}}
    \cventry{Postdoctoral Researcher}{Eberhard Karls University of Tübingen}{Tübingen, Germany}{Sept 2019--Dec 2019}{\begin{cvitems}
\item DFG Center for Advanced Studies “Words, Bones, Genes, Tools”  \href{https://uni-tuebingen.de/forschung/forschungsschwerpunkte/forschungsgruppen/words-bones-genes-tools/}{\faExternalLink}
\item Exploring geometric morphometrics methods in lithic analysis
\end{cvitems}}
    \cventry{Research Assistant}{Heidelberg Academy of Sciences and Humanities}{Tübingen, Germany}{April 2019--Aug 2019}{\begin{cvitems}
\item The Role of Culture in Early Expansions of Humans (ROCEEH)  \href{https://www.hadw-bw.de/en/research/research-center/roceeh}{\faExternalLink}
\item Synthesizing archaeological data into the ROCEEH Out of Africa Database (ROAD)
\end{cvitems}}
    \cventry{Visiting Research Fellow}{Hebrew University of Jerusalem}{Jerusalem, Israel}{April 2015--July 2015}{\begin{cvitems}
\item Recipient of the research grant awarded by Fondazione Atlante (Italy)
\item Research on the Early Ahmarian and Levantine Aurignacian from Kebara and Hayonim under the supervision of Prof. Anna Belfer-Cohen
\end{cvitems}}
\end{cventries}

\hypertarget{education}{%
\section{Education}\label{education}}

\begin{cventries}
    \cventry{Ph.D. in Paleolithic Archaeology}{Eberhard Karls University of Tübingen}{Tübingen, Germany}{Dec 2015--Jan 2019}{\begin{cvitems}
\item Dissertation: A critical assessment of the Aurignacian: Insights from Fumane Cave in northern Italy
\item Supervisors: Prof. Nicholas J. Conard, PhD and Prof. Dr. Michael Bolus
\item Grade: magna cum Laude
\end{cvitems}}
    \cventry{M.A. in Quaternary, Prehistory, and Archaeology}{University of Ferrara}{Ferrara, Italy}{Nov 2013--July 2015}{\begin{cvitems}
\item Dissertation: Morpho-metric variability of Protoaurignacian retouched bladelets. The cases of Grotta di Fumane, Grotte d’Isturitz and Grotte des Cottés
\item Supervisors: Prof. Marco Peresani and Prof. Marie Soressi (Leiden University, Netherlands)
\item Grade: 110 cum laude
\end{cvitems}}
    \cventry{Erasmus+ Exchange Programme}{University of Toulouse Jean Jaurès}{Toulouse, France}{Sept 2014--Feb 2015}{\begin{cvitems}
\item Undertook courses in Paleolithic archaeology and lithic technology
\item Carried out research on the lithic assemblages from Isturitz at TRACES (Travaux et Recherches Archéologiques sur les Cultures, les Espaces et les Sociétés, UMR 5608)
\item Program supervisors: Prof. François Bon and Dr. Nicholas Teyssandier
\end{cvitems}}
    \cventry{B.A. in Cultural Heritage Sciences}{University of Rome Tor Vergata}{Rome, Italy}{Oct 2010--Oct 2013}{\begin{cvitems}
\item Dissertation: The last Neanderthal: transition and extinction between Middle and Upper Paleolithic in the Iberian Peninsula
\item Supervisor: Prof. Mario Federico Rolfo
\item Grade: 110 cum laude
\end{cvitems}}
    \cventry{Erasmus Exchange Programme}{University of Granada}{Granada, Spain}{Sept 2012--July 2013}{\begin{cvitems}
\item Undertook courses in archaeology and history
\item Conducted bibliographic research on the Middle-to-Upper Paleolithic transition in the Iberian Peninsula
\item Program supervisor: Prof. Juan Antonio Cámara Serrano
\end{cvitems}}
\end{cventries}

\hypertarget{peer-reviewed-publications}{%
\section{Peer-Reviewed Publications}\label{peer-reviewed-publications}}

\small
\setlength{\leftskip}{0cm}

\textbf{2023}

\setlength{\leftskip}{1cm}

Lombao, D.,\textbf{Falcucci, A.}, Moos, E., Peresani, M. (2023)
Unravelling technological behaviors through core reduction intensity.
The case of the early Protoaurignacian assemblage from Fumane Cave.
\textbf{Journal of Archaeological Science}, 160:105889.
\url{https://doi.org/10.1016/j.jas.2023.105889} \emph{(co-first
authorship)}

Rossini, M., \textbf{Falcucci, A.}, Dominici, C., Ronchitelli, A.,
Tomasso, A., Boschin, F. (2023) Application of 2D shape analysis to
study Epigravettian lithic assemblages: assessing its analytical
potential. \textbf{Acta IMEKO}, 12(4):1-8.
\url{https://doi.org/10.21014/actaimeko.v12i4.1539}

\setlength{\leftskip}{0cm}

\textbf{2022}

\setlength{\leftskip}{1cm}

\textbf{Falcucci, A.}, Karakostis, F.A., Göldner, D., Peresani, M.
(2022) Bringing shape into focus: Assessing differences between blades
and bladelets and their technological significance in 3D form.
\textbf{Journal of Archaeological Science: Reports}, 43:103490.
\url{https://doi.org/10.1016/j.jasrep.2022.103490}

\textbf{Falcucci, A.}, Peresani, M. (2022) The contribution of
integrated 3D model analysis to Protoaurignacian stone tool design.
\textbf{PLoS One}, 17(5):e0268539.
\url{https://doi.org/10.1371/journal.pone.0268539}

Göldner, D., Karakostis, F.A., \textbf{Falcucci, A.} (2022) Practical
and technical aspects for the 3D scanning of lithic artefacts using
micro-computed tomography techniques and laser light scanners for
subsequent geometric morphometric analysis. Introducing the StyroStone
protocol. \textbf{PLoS One}, 17(4):e0267163.
\url{https://doi.org/10.1371/journal.pone.0267163}

Rossini, M., \textbf{Falcucci, A.}, Dominici, C., Ronchitelli, A.,
Tomasso, A., Boschin, F. (2022) Analytical potential of 2D shape
analysis to study Epigravettian lithic assemblages. \textbf{IMEKO TC-4}.
\url{https://doi.org/10.21014/tc4-ARC-2022.011}

\setlength{\leftskip}{0cm}

\textbf{2021}

\setlength{\leftskip}{1cm}

Aleo, A., Duches, R., \textbf{Falcucci, A.}, Rots, V., Peresani, M.
(2021) Scraping hide in the early Upper Paleolithic: Insights into the
life and function of the Protoaurignacian endscrapers at Fumane Cave.
\textbf{Archaeological and Anthropological Sciences}, 13(8):137.
\url{https://doi.org/10.1007/s12520-021-01367-4}

\setlength{\leftskip}{0cm}

\textbf{2020}

\setlength{\leftskip}{1cm}

Bataille, G., \textbf{Falcucci, A.}, Tafelmaier, Y., Conard, N. J.
(2020). Technological differences between Kostenki 17/II (Spitsynskaya
industry, Central Russia) and the Protoaurignacian. Reply to Dinnis et
al.~(2019). \textbf{Journal of Human Evolution}, 146:102685.
\url{https://doi.org/10.1016/j.jhevol.2019.102685}

\textbf{Falcucci, A.}, Conard, N.J., Peresani, M. (2020) Breaking
through the Aquitaine frame: A re-evaluation on the significance of
regional variants during the Aurignacian as seen from a key record in
southern Europe. \textbf{Journal of Anthropological Sciences},
98:99-140. \url{https://doi.org/10.4436/JASS.98021}

Marciani, G., Ronchitelli, A., Arrighi, S., Badino, F., Bortolini, E.,
Boscato, P., Boschin, F., Crezzini, J., Delpiano, D., \textbf{Falcucci,
A.}, Figus, C., Lugli, F., Negrino, F., Oxilia, G., Romandini, M.,
Riel-Salvatore, J., Spinapolice, E., Peresani, M., Moroni, A., Benazzi,
S. (2020) Lithic techno-complexes in Italy from 50 to 39 thousand years
BP: an overview of cultural and technological changes across the
Middle-Upper Palaeolithic boundary. \textbf{Quaternary International},
551:123-149. \url{https://doi.org/10.1016/j.quaint.2019.11.005}

\setlength{\leftskip}{0cm}

\textbf{2019}

\setlength{\leftskip}{1cm}

\textbf{Falcucci, A.} \& Peresani, M. (2019) A pre-Heinrich Event 3
assemblage at Fumane Cave and its contribution for understanding the
beginning of the Gravettian in Italy. \textbf{Quartär}, 66:135-154.
\url{https://doi.org/10.7485/QU66_6}

Ioannidou, M., \textbf{Falcucci, A.}, Röding, C., Kandel, A.W. (2019)
Eighth Annual Meeting of the European Society for the Study of Human
Evolution. \textbf{Evolutionary Anthropology Issues News and Reviews},
28(2):52--54. \url{https://doi.org/10.1002/evan.21770}

\setlength{\leftskip}{0cm}

\textbf{2018}

\setlength{\leftskip}{1cm}

Caricola, I., Zupancich, A., Moscone, D., Mutri, G., \textbf{Falcucci,
A.}, Duches, R., Peresani, M., Cristiani, E. (2018) An integrated method
for understanding the function of macro-lithic tools. Use wear, 3D and
spatial analyses of an Early Upper Palaeolithic assemblage from North
Eastern Italy. \textbf{PLoS One}, 13:e0207773.
\url{https://doi.org/10.1371/journal.pone.0207773}

\textbf{Falcucci, A.} (2018) Towards a renewed definition of the
Protoaurignacian. \textbf{Mitteilungen der Gesellschaft für
Urgeschichte}, 27:87--130.

\textbf{Falcucci, A.} \& Peresani, M. (2018) Protoaurignacian Core
Reduction Procedures: Blade and Bladelet Technologies at Fumane Cave.
\textbf{Lithic Technology}, 43:125-140.
\url{https://doi.org/10.1080/01977261.2018.1439681}

\textbf{Falcucci, A.}, Peresani, M., Roussel, M., Normand, C., Soressi,
M. (2018) What's the point? Retouched bladelet variability in the
Protoaurignacian. Results from Fumane, Isturitz, and Les Cottés.
\textbf{Archaeological and Anthropological Sciences}, 10:539--554.
\url{https://doi.org/10.1007/s12520-016-0365-5}

\setlength{\leftskip}{0cm}

\textbf{2017}

\setlength{\leftskip}{1cm}

\textbf{Falcucci, A.}, Conard, N. J., Peresani, M. (2017) A critical
assessment of the Protoaurignacian lithic technology at Fumane Cave and
its implications for the definition of the earliest Aurignacian.
\textbf{PLoS One}, 12(12): e0189241.
\url{https://doi.org/10.1371/journal.pone.0189241}

\setlength{\leftskip}{0cm}

\hypertarget{non-refereed-publications}{%
\section{Non-Refereed Publications}\label{non-refereed-publications}}

\textbf{Falcucci, A.} (2019) The importance of re-evaluating the
Aurignacian industry for understanding the Early Upper Paleolithic. 15th
Newsletter of The Role of Culture in Early Expansions of Humans
(ROCEEH).

\setlength{\leftskip}{0cm}

\hypertarget{d-scanning-protocols}{%
\section{3D-Scanning Protocols}\label{d-scanning-protocols}}

\textbf{Falcucci, A.}, (2022) MicroStone: Exploring the capabilities of
the Artec Micro in scanning stone tools. \textbf{protocols.io}.
dx.doi.org/10.17504/protocols.io.81wgb6781lpk/v1

Göldner, D., Karakostis, F. A., \textbf{Falcucci, A.}, (2022)
StyroStone: A protocol for scanning and extracting three-dimensional
meshes of stone artefacts using Micro-CT scanners V.3.
\textbf{protocols.io}. dx.doi.org/10.17504/protocols.io.4r3l24d9qg1y/v2

\setlength{\leftskip}{0cm}

\hypertarget{open-access-repositories}{%
\section{Open Access Repositories}\label{open-access-repositories}}

\textbf{Falcucci, A.} (2022) Research compendium for `The contribution
of integrated 3D model analysis to Protoaurignacian stone tool design'.
\textbf{Zenodo}. \url{doi:10.5281/zenodo.6504415}

\textbf{Falcucci, A.}, Karakostis, F. A., Göldner, D., Peresani, M.
(2022) Research compendium for `Bringing shape into focus: Assessing
differences between blades and bladelets and their technological
significance in 3D form'. \textbf{Zenodo}.
\url{doi:10.5281/zenodo.6368200}

\textbf{Falcucci, A.} and Peresani, M. (2022) The Open Aurignacian
Project. Volume 1: Fumane Cave in northeastern Italy. \textbf{Zenodo}.
\url{doi:10.5281/zenodo.6362149}

Göldner, D., Karakostis, F. A., \textbf{Falcucci, A.}, (2022) Research
compendium for `Practical and technical aspects for the 3D scanning of
lithic artefacts using micro-computed tomography techniques and laser
light scanners for subsequent geometric morphometric analysis.
Introducing the StyroStone protocol'. \textbf{Zenodo}.
\url{doi:10.5281/zenodo.6365680}

Lombao, D.,\textbf{Falcucci, A.}, Moos, E., Peresani, M. (2023) Research
compendium for `Unravelling technological behaviors through core
reduction intensity. The case of the early Protoaurignacian assemblage
from Fumane Cave'. \textbf{Zenodo}.
\url{https://doi.org/10.5281/zenodo.8212572}

\setlength{\leftskip}{0cm}

\hypertarget{grants-and-awards}{%
\section{Grants and Awards}\label{grants-and-awards}}

\normalsize

Total awarded \textasciitilde420,000€

\begin{cventries}
    \cventry{DFG, German Research Foundation (Bonn, DE)}{Research Grant}{24,400€}{July 2023}{\begin{cvitems}
\item Awarded for a three-month extension of the project “Investigating Early Upper Paleolithic Technological Variability and Cultural Dynamics South of the Alps”
\end{cvitems}}
    \cventry{Reinhard-Frank-Stiftung (Hamburg, DE)}{Research Grant}{15,000€}{Dec 2022}{\begin{cvitems}
\item Awarded for a research stay at Harvard University (Cambridge, Massachusetts, USA)
\end{cvitems}}
    \cventry{Open Access Publication Fund (Tübingen, DE)}{Publication Grant}{1,200€}{May 2022}{\begin{cvitems}
\item Awarded to cover the publication fees of open access journals
\end{cvitems}}
    \cventry{Open Access Publication Fund (Tübingen, DE)}{Publication Grant}{770€}{April 2022}{\begin{cvitems}
\item Awarded to cover the publication fees of open access journals
\end{cvitems}}
    \cventry{DFG, German Research Foundation (Bonn, DE)}{Research Grant}{316,229€}{Sept 2020}{\begin{cvitems}
\item Awarded for the project “Investigating Early Upper Paleolithic Technological Variability and Cultural Dynamics South of the Alps”
\end{cvitems}}
    \cventry{University of Tübingen (Tübingen, DE)}{Research Grant}{1,000€}{Jan 2019}{\begin{cvitems}
\item Awarded for conducting research at the Department  of Geosciences, Early Prehistory and Quaternary Ecology working group
\end{cvitems}}
    \cventry{University of Ferrara (Ferrara, IT)}{Research Grant}{3,000€}{March 2019}{\begin{cvitems}
\item Awarded for analyzing the Gravettian lithic assemblage from Fumane Cave
\end{cvitems}}
    \cventry{European Society for the study of Human Evolution (ESHE)}{Travel Bursary}{150€}{Sept 2018}{\begin{cvitems}
\item Awarded to present a paper at the ESHE annual meeting in Faro, Portugal
\end{cvitems}}
    \cventry{International Union of Prehistoric and Protohistoric Sciences (UISPP)}{Travel Bursary}{200€}{Aug 2018}{\begin{cvitems}
\item Awarded  to present a paper at UISPP conference in Paris, France
\end{cvitems}}
    \cventry{Universitätsbund Tübingen e. V. (Tübingen, DE)}{Travel Bursary}{150€}{Aug 2018}{\begin{cvitems}
\item Awarded to present my Ph.D. findings at the UISPP conference in Paris, France
\end{cvitems}}
    \cventry{Open Access Publication Fund (Tübingen, DE)}{Publication Grant}{1,150€}{Dec 2017}{\begin{cvitems}
\item Awarded to cover the publication fees of open access journals
\end{cvitems}}
    \cventry{European Society for the study of Human Evolution (ESHE)}{Travel Bursary}{250€}{Sept 2016}{\begin{cvitems}
\item Awarded to present a paper at the ESHE annual meeting in Madrid, Spain
\end{cvitems}}
    \cventry{Baden-Württemberg Ministry of Science, Research and Art (DE)}{Ph.D. Scholarship}{49,200€}{Dec 2015}{\begin{cvitems}
\item Awarded to conduct a doctoral research project at the University of Tübingen
\end{cvitems}}
    \cventry{Fondazione Atlante (Milan, Italy)}{Research Grant}{4,000€}{April 2015}{\begin{cvitems}
\item Awarded for a research stay at the Hebrew University of Jerusalem, Israel
\end{cvitems}}
    \cventry{European Commission}{Erasmus+ Scholarship}{3,000€}{Sept 2014}{\begin{cvitems}
\item Study abroad program at the University of Toulouse, France
\end{cvitems}}
    \cventry{International Union of Prehistoric and Protohistoric Sciences (UISPP)}{Travel Bursary}{350€}{Aug 2014}{\begin{cvitems}
\item Awarded to present a paper at UISPP conference in Burgos, Spain
\end{cvitems}}
    \cventry{European Commission}{Erasmus Scholarship}{4,500€}{Sept 2012}{\begin{cvitems}
\item Study abroad program at the University of Granada, Spain
\end{cvitems}}
\end{cventries}

\small
\setlength{\leftskip}{0cm}

\hypertarget{teaching-experience-and-supervision}{%
\section{Teaching Experience and
Supervision}\label{teaching-experience-and-supervision}}

I teach prehistoric archaeology, lithic technology, and 3D-based methods
in the framework of the MSc program \textbf{Archaeological Sciences and
Human Evolution}
\href{https://uni-tuebingen.de/en/study/finding-a-course/degree-programs-available/detail/course/archaeological-sciences-and-human-evolution-master/}{\faExternalLink}
at the \textbf{Institute for Archaeological Sciences}
\href{https://uni-tuebingen.de/en/faculties/faculty-of-science/departments/geosciences/work-groups-contacts/prehistory-and-archaeological-sciences/ina/}{\faExternalLink}
(University of Tübingen, Germany).

\setlength{\leftskip}{0cm}

\textbf{Winter Term 2023/2024}

\setlength{\leftskip}{1cm}

UFG-BA, UFG-MA, NWA: \emph{Prehistory of Mediterranean Europe: late
Neanderthals and modern humans} (3 ECTS).

\setlength{\leftskip}{0cm}

\textbf{Summer Term 2023}

\setlength{\leftskip}{1cm}

ASHE 6f-2, ASHE 5, UFGAM\_MA\_6.3: \emph{3D Applications in lithic
analysis} (3 ECTS).

ASHE-6f-1: \emph{Stone Age Technology} (3 ECTS). Course held in
collaboration with Prof.~N.J. Conard.

\setlength{\leftskip}{0cm}

\textbf{Summer Term 2022}

\setlength{\leftskip}{1cm}

UFG-BA, UFG-MA, NWA: \emph{Prehistory of Mediterranean Europe: late
Neanderthals and modern humans} (3 ECTS).

ASHE 6f-2, ASHE 5, UFGAM\_MA\_6.3: \emph{3D Applications in lithic
analysis} (3 ECTS).

ASHE-6f-1: \emph{Stone Age Technology} (3 ECTS). Course held in
collaboration with other colleagues from the University of Tübingen.

\setlength{\leftskip}{0cm}

\textbf{Summer Term 2021}

\setlength{\leftskip}{1cm}

UFG-BA, UFG-MA, NWA: \emph{Prehistory of Mediterranean Europe: late
Neanderthals and modern humans} (3 ECTS).

\setlength{\leftskip}{0cm}

\textbf{Supervision}

\setlength{\leftskip}{1cm}

I supervised \textbf{seven} research assistants (\emph{Wissenschaftliche
Hilfskraft}) in the framework of my DFG project ``Investigating Early
Upper Paleolithic Technological Variability and Cultural Dynamics South
of the Alps'' to assist in lithic analysis, experimental activity, and
3D-scanning.

\setlength{\leftskip}{0cm}

\hypertarget{technical-skills}{%
\section{Technical Skills}\label{technical-skills}}

\begin{table}[!h]
\centering\begingroup\fontsize{9}{11}\selectfont

\begin{tabular}{>{\centering\arraybackslash}p{4.75cm}>{\centering\arraybackslash}p{4.75cm}>{\centering\arraybackslash}p{4.75cm}}
\toprule
\textcolor[HTML]{414141}{\textbf{Coding Languages}} & \textcolor[HTML]{414141}{\textbf{Software}} & \textcolor[HTML]{414141}{\textbf{3D Scanners}}\\
\midrule
\textcolor[HTML]{7f7f7f}{R, RMarkdown, RStudio} & \textcolor[HTML]{7f7f7f}{Adobe CreativeCloud, Avizo, Geomagic Wrap, GitHub, IBM SPSS, Meshlab, Microsoft Office, Past} & \textcolor[HTML]{7f7f7f}{Artec Micro, Artec Space Spider, EinScan Pro HD}\\
\bottomrule
\end{tabular}
\endgroup{}
\end{table}

\hypertarget{additional-training-and-professional-development}{%
\section{Additional Training and Professional
Development}\label{additional-training-and-professional-development}}

\begin{cvhonors}
    \cvhonor{}{\textbf{Certificate Science Communication and Media Competence} (Graduate Academy, Tübingen)}{}{2023}
    \cvhonor{}{\textbf{Data Manipulation with R Tidyverse} (TransmittingScience, Barcelona)}{}{2022}
    \cvhonor{}{\textbf{R's ggplot2} (TransmittingScience, Barcelona)}{}{2022}
    \cvhonor{}{\textbf{R Programming Course} (TransmittingScience, Barcelona)}{}{2022}
    \cvhonor{}{\textbf{Applying for a Marie Curie Postdoctoral Fellowship} (YellowResearch, Amsterdam)}{}{2022}
    \cvhonor{}{\textbf{Strategic Career Planning in Academia} (Graduate Academy, Tübingen)}{}{2022}
    \cvhonor{}{\textbf{Taking the Lead for Postdocs} (Graduate Academy, Tübingen)}{}{2022}
    \cvhonor{}{\textbf{Advanced Agile Project Management for Researchers} (Graduate Academy, Tübingen)}{}{2022}
    \cvhonor{}{\textbf{Principles of Grant Writing} (Graduate Academy, Tübingen)}{}{2021}
\end{cvhonors}

\hypertarget{professional-memberships}{%
\section{Professional Memberships}\label{professional-memberships}}

\begin{cvhonors}
    \cvhonor{}{\textbf{Computer Applications and Quantitative Methods in Archaeology (CAA)}}{}{Since 2022}
    \cvhonor{}{\textbf{Paleoanthropology Society}}{}{Since 2023}
    \cvhonor{}{\textbf{Society for American Archaeology (SAA)}}{}{Since 2022}
    \cvhonor{}{\textbf{Gesellschaft für Urgeschichte e.V. (GFU)}}{}{Since 2020}
    \cvhonor{}{\textbf{Hugo Obermaier-Gesellschaft (HOG)}}{}{Since 2016}
    \cvhonor{}{\textbf{European Society for the Study of Human Evolution (ESHE)}}{}{Since 2016}
\end{cvhonors}

\setlength{\leftskip}{0cm}

\hypertarget{academic-service}{%
\section{Academic Service}\label{academic-service}}

I have served as academic reviewer for several peer-reviewed journals
such as \textbf{PLoS ONE} (4), \textbf{Archeological and Anthropological
Sciences} (1), \textbf{Journal of Archaeological Science: Reports} (1),
\textbf{Journal of Paleolithic Archaeology} (3), \textbf{IIPP Journal}
(2), as well as for several edited book chapters.

\textbf{Member of doctoral committee}

I was an external \textbf{member of the doctoral committee} for the
dissertation entitled ``Reducción y gestión volumétrica: aproximación a
la variabilidad y evolución de las dinámicas de explotación durante el
Pleistoceno inferior y medio europeo, a través de los conjuntos de Gran
Dolina y Galería (Sierra de Atapuerca, Burgos) y de El Barranc de la
Boella (La Canonja, Tarragona)'', defended by \textbf{Dr.~Diego Lombao
Vázquez} at the Universitat Rovira i Virgili (Tarragona, Spain) on the
\textbf{20.12.2021}

\newpage
\setlength{\leftskip}{0cm}

\hypertarget{organization-of-conference-sessions}{%
\section{Organization of Conference
Sessions}\label{organization-of-conference-sessions}}

\setlength{\leftskip}{0cm}

\textbf{2023}

\setlength{\leftskip}{1cm}

Lombao, D., Rabuñal, J.R., \textbf{Falcucci, A.}, Lebret, J.B.,
Tobalina-Pulido, L. (2023) Replicable Archaeology: Looking for Workflows
and Data Management Strategies Fostering Data Reuse and Methodological
Transferability in Archaeological Science. 29th Annual Meeting of the
\textbf{European Association of Archaeologists (EAA)} (Belfast, Northern
Ireland). \emph{Session \#603}.

\setlength{\leftskip}{0cm}

\hypertarget{scholarly-presentations}{%
\section{Scholarly Presentations}\label{scholarly-presentations}}

\setlength{\leftskip}{0cm}

\textbf{2023}

\setlength{\leftskip}{1cm}

\textbf{Falcucci, A.} (2023) Searching for solid and replicable
frameworks in stone tool analysis: A discussion surrounding recent
achievements and standing challenges. \textbf{Advances in Human
Evolution, Adaptation \& Diversity (AHEAD)} (Tarragona, Spain).
\emph{Podium Presentation}.

\textbf{Falcucci, A.} \& Lombao, D. (2023) Measuring reduction intensity
in laminar cores: an experimental approach and archaeological
application. 88th Annual Meeting of the \textbf{Society for American
Archaeology (SAA)} (Portland, USA). \emph{Podium Presentation}.

\textbf{Falcucci, A.}, Moroni, A., Negrino, F., Peresani, M.,
Riel-Salvatore, J., Ronchitelli, A. (2023) Investigating Aurignacian
technological variability and population connectedness south of the Alps
and along peninsular Italy. Annual \textbf{Paleoanthropology Society}
Meeting (Portland, USA). \emph{Podium Presentation}.

\textbf{Falcucci, A.} \& Pargeter, J. (2023) Looking for an efficient
workflow in the R programming language for replicable lithic
quantitative analysis. 29th Annual Meeting of the \textbf{European
Association of Archaeologists (EAA)} (Belfast, Northern Ireland).
\emph{Podium Presentation}.

Lombao, D. \& \textbf{Falcucci, A.} (2023) Blade and bladelet cores in
the Protoaurignacian: A new method for measuring reduction intensity.
64th Annual Meeting of the \textbf{Hugo Obermaier Society} (Aarhus,
Denmark). \emph{Poster Presentation}.

\setlength{\leftskip}{0cm}

\textbf{2022}

\setlength{\leftskip}{1cm}

\textbf{Falcucci, A.} (2022) Integrating 2D and 3D shape analysis in
lithic technology: A discussion around a case study and future research
perspectives. 49th Meeting of the \textbf{Computer Applications and
Quantitative Methods in Archaeology} (Oxford, UK). \emph{Podium
Presentation}.

\textbf{Falcucci, A.}, Giusti, D., Zangrossi, F., De Lorenzi, M.,
Ceregatti, L., Peresani, M. (2022). Linking blades: a systematic
refitting analysis of blade fragments from the Protoaurignacian sequence
of Fumane Cave. 12th Annual Meeting of the \textbf{European Society for
the study of Human Evolution} (Tübingen, DE). \emph{Poster
Presentation}.

\textbf{Falcucci, A.}, Karakostis, F. A., Göldner, D., \& Peresani, M.
(2022) Identifying and quantifying morphological separation between
blade and bladelet productions through 3D shape analysis. 87th Annual
Meeting of the \textbf{Society for American Archaeology (SAA)} (Chicago,
USA). \emph{Podium Presentation}.

Rossini, M., \textbf{Falcucci, A.}, Dominici, C, Ronchitelli, A.,
Tomasso, A., Boschin, F. (2022) Analytical potential of 2D shape
analysis to study Epigravettian lithic assemblages. 4th IMEKO
\textbf{International Conference on Metrology for Archaeology and
Cultural Heritage} (Cosenza, Italy). \emph{Podium Presentation}.

\setlength{\leftskip}{0cm}

\textbf{2021}

\setlength{\leftskip}{1cm}

\textbf{Falcucci, A.}, Karakostis, F. A., Göldner, D., \& Peresani, M.
(2021) Identifying and quantifying morphological separation between
blade and bladelet productions through 3D shape analysis. 13th
\textbf{International Symposium of Knappable Materials} (Online).
\emph{Podium Presentation}.

\textbf{Falcucci, A.}, and Peresani, M. (2021) Assessing the Gravettian
evidence from Fumane Cave. 62nd Annual Meeting of the \textbf{Hugo
Obermaier Society} for Quaternary Research and Archaeology of the Stone
Age. \emph{Podium Presentation}.

\setlength{\leftskip}{0cm}

\textbf{2019}

\setlength{\leftskip}{1cm}

\textbf{Falcucci, A.} (2019) The site of Fumane Cave in northern Italy.
Formation processes, stratigraphy, and dating of the Protoaurignacian.
Archéologie Alsace - ÄUQÖ Tübingen \textbf{CIERA Meeting} (Tübingen,
DE). \emph{Podium Presentation}.

\textbf{Falcucci, A.}, Conard, N. J., Peresani, M. (2019) A
re-evaluation of the Protoaurignacian sequence at Fumane Cave in
northern Italy. 61st Annual Meeting of the \textbf{Hugo Obermaier
Society} for Quaternary Research and Archaeology of the Stone Age
(Erkrath, DE). \emph{Podium Presentation}.

\setlength{\leftskip}{0cm}

\textbf{2018}

\setlength{\leftskip}{1cm}

Aureli, D., Arrighi, S., Bortolini, E., Boscato, P., Boschin, F.,
Crezzini, J., \textbf{Falcucci, A.}, Delpiano, D., Figus, C., Moroni,
A., Negrino, F., Peresani, M., Riel-Salvatore, J., Romandini, M.,
Ronchitelli, A., Spinapolice, E., Benazzi, S. (2018) Italian Peninsula
between 45 and 39 ky ago: the sunset of the ``old'' and the dawn of the
``new''? Let the lithic industries tell!XVIII \textbf{UISPP World
Congress} (Paris, France). \emph{Podium Presentation}.

\textbf{Falcucci, A.}, Conard, N. J., Peresani, M. (2018) Testing
technological definitions: a critical assessment of the Protoaurignacian
at Fumane Cave. XVIII \textbf{UISPP World Congress} (Paris, France).
\emph{Podium Presentation}.

\textbf{Falcucci, A.}, Conard, N. J., Peresani, M. (2018) The
chrono-cultural narrative of the Fumanian Aurignacian supports the
inapplicability of the Aquitaine Model on a supra-regional scale. 8th
Annual Meeting of the \textbf{European Society for the study of Human
Evolution} (Faro, PT). \emph{Podium Presentation}.

\setlength{\leftskip}{0cm}

\textbf{2017}

\setlength{\leftskip}{1cm}

Bataille, G., Bolus, M., Conard, N. J., \textbf{Falcucci, A.}, Peresani,
M., Tafelmaier, Y. (2017) The techno-typological variability of the
European Aurignacian from a multi-regional and diachronic perspective.
59st Annual Meeting of the \textbf{Hugo Obermaier Society} for
Quaternary Research and Archaeology of the Stone Age (Aurich, DE).
\emph{Podium Presentation}.

Peresani, M., Delpiano, D., Aleo, A., Duches, R., \textbf{Falcucci, A.},
Forte, M., Gennai, J., Nannini, N., Romandini, M., Terlato, G. (2017) La
transizione Neanderthal -- Sapiens in Italia Nord-orientale: Ricerche
dell'università di Ferrara. Annual Meeting of the \textbf{TourismA
Society} for Archaeology and Tourism (Firenze, IT). \emph{Poster
Presentation}.

\setlength{\leftskip}{0cm}

\textbf{2016}

\setlength{\leftskip}{1cm}

\textbf{Falcucci, A.} \& Peresani, M. (2016) New investigations on the
Protoaurignacian lithic technology of Fumane Cave. 6th Annual Meeting of
the \textbf{European Society for the study of Human Evolution} (Madrid,
ES). \emph{Poster Presentation}.

\textbf{Falcucci, A.}, Peresani, M., Roussel, M., Normand, C., Soressi,
M. (2016) Protoaurignacian retouched bladelets: where do we stand? 59st
Annual Meeting of the \textbf{Hugo Obermaier Society} for Quaternary
Research and Archaeology of the Stone Age (Budapest, HU). \emph{Poster
Presentation}.

\setlength{\leftskip}{0cm}

\textbf{2014}

\setlength{\leftskip}{1cm}

Romandini, M., \textbf{Falcucci., A.}, Gurioli, F., Broglio, A. (2014)
First European Anatomically Modern Humans and the ``route of shells'':
chronological and qualitative evidence from the Veneto plain region
sites. XVII \textbf{UISPP World Congress} (Burgos, ES). \emph{Poster
Presentation}.

\setlength{\leftskip}{0cm}

\hypertarget{invited-academic-lectures}{%
\section{Invited Academic Lectures}\label{invited-academic-lectures}}

\setlength{\leftskip}{0cm}

\textbf{2023}

\setlength{\leftskip}{1cm}

The Institute of Archaeology, \textbf{Hebrew University of Jerusalem}.
\emph{Title}: Bridging past behavior and modern technologies:
Three-dimensional and experimental insights into the Aurignacian along
peninsular Italy.

Department of Anthropology, \textbf{Harvard University}. \emph{Title}:
3D visualization and analysis in lithic studies.

\setlength{\leftskip}{0cm}

\textbf{2022}

\setlength{\leftskip}{1cm}

African Paleosciences Laboratory, Department of Anthropology,
\textbf{New York University}. \emph{Title}: A discussion surrounding 3D
scanning technology for archaeological sciences.

\setlength{\leftskip}{0cm}

\textbf{2020}

\setlength{\leftskip}{1cm}

Institut für Ur- und Frühgeschichte und Archäologie des Mittelalters,
\textbf{University of Tübingen}. Kolloquium Ältere Urgeschichte.
\emph{Title}: A pre-Heinrich Event 3 assemblage at Fumane Cave,
Northeast Italy.

\setlength{\leftskip}{0cm}

\textbf{2019}

\setlength{\leftskip}{1cm}

Institut für Ur- und Frühgeschichte, \textbf{University of Cologne}.
Prähistorisches Kolloquium. \emph{Title}: Breaking through the
Aurignacian frame: A re-evaluation of the archaeological record south of
the Alps.

DFG Center for Advanced Studies ``Words, Bones, Genes, Tools: Tracking
Linguistic, Cultural and Biological Trajectories of the Human Past'',
\textbf{University of Tübingen}. Center Colloquium. \emph{Title}: The
first European archers? Review of Sano et al.~2019 (The earliest
evidence for mechanically delivered projectile weapons in Europe, Nature
Ecology and Evolution).

\setlength{\leftskip}{0cm}

\textbf{2018}

\setlength{\leftskip}{1cm}

Institut für Ur- und Frühgeschichte und Archäologie des Mittelalters,
\textbf{University of Tübingen}. Kolloquium Ältere Urgeschichte.
\emph{Title}: A Chrono-cultural Narrative of the Fumanian Aurignacian.

\setlength{\leftskip}{0cm}

\textbf{2016}

\setlength{\leftskip}{1cm}

Dipartimento di Studi Umanistici, Sezione di Scienze Preistoriche e
Antropologiche, \textbf{University of Ferrara}. Student seminar.
\emph{Title}: L'Aurignaziano di Grotta di Fumane.

\newpage
\setlength{\leftskip}{0cm}

\textbf{2015}

\setlength{\leftskip}{1cm}

Institut für Naturwissenschaftliche Archäologie, \textbf{University of
Tübingen}. INA Colloquium. \emph{Title}: The Protoaurignacian in the
context of the Early Upper Paleolithic. Concepts, variability and
possible technological convergences with the Early Ahmarian.

\setlength{\leftskip}{0cm}

\hypertarget{fieldwork-experience}{%
\section{Fieldwork Experience}\label{fieldwork-experience}}

I have taken part in several \textbf{archaeological excavations} across
France, Germany, Israel, and Italy. Likewise, I have spent several
months in many universities and research institutions to analyze Upper
Paleolithic \textbf{lithic assemblages} from France (Isturitz, Les
Cottés), Israel (Hayonim, Kebara, Manot Cave), Italy (Bombrini,
Castelcivita, Fumane, Serino, Villabruna), and Lebanon (Ksar Akil).

\hypertarget{languages}{%
\section{Languages}\label{languages}}

\begin{table}[H]
\centering\begingroup\fontsize{9}{11}\selectfont

\begin{tabular}{>{\centering\arraybackslash}p{2.4cm}>{\centering\arraybackslash}p{2.4cm}>{\centering\arraybackslash}p{2.4cm}>{\centering\arraybackslash}p{2.4cm}>{\centering\arraybackslash}p{2.4cm}>{\centering\arraybackslash}p{2.4cm}}
\toprule
Skill & Italian & English & Spanish & French & German\\
\midrule
Reading & \textcolor[HTML]{009acd}{Native} & \textcolor[HTML]{009acd}{C2} & \textcolor[HTML]{009acd}{C2} & \textcolor[HTML]{009acd}{C1} & \textcolor[HTML]{009acd}{B2}\\
Writing & \textcolor[HTML]{009acd}{Native} & \textcolor[HTML]{009acd}{C1} & \textcolor[HTML]{009acd}{C1} & \textcolor[HTML]{009acd}{B2} & \textcolor[HTML]{009acd}{B1}\\
Listening & \textcolor[HTML]{009acd}{Native} & \textcolor[HTML]{009acd}{C2} & \textcolor[HTML]{009acd}{C2} & \textcolor[HTML]{009acd}{C1} & \textcolor[HTML]{009acd}{B2}\\
Speaking & \textcolor[HTML]{009acd}{Native} & \textcolor[HTML]{009acd}{C1} & \textcolor[HTML]{009acd}{C1} & \textcolor[HTML]{009acd}{B2} & \textcolor[HTML]{009acd}{B1}\\
\bottomrule
\multicolumn{6}{l}{\rule{0pt}{1em}\textit{ } \scriptsize Common European Framework of Reference for Languages: A1/A2: Basic User. B1/B2: Independent User. C1/C2: Proficient User}\\
\end{tabular}
\endgroup{}
\end{table}

\vfill
\footnotesize

\textbf{Note:} this is an R Markdown CV created with the package
\textbf{\emph{vitae (v. 0.5.4)}} by O'Hara-Wild and Hyndman (2022).
\url{https://CRAN.R-project.org/package=vitae}.


\label{LastPage}~
\end{document}
